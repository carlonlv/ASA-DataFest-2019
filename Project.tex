\documentclass[]{article}
\usepackage{lmodern}
\usepackage{amssymb,amsmath}
\usepackage{ifxetex,ifluatex}
\usepackage{fixltx2e} % provides \textsubscript
\ifnum 0\ifxetex 1\fi\ifluatex 1\fi=0 % if pdftex
  \usepackage[T1]{fontenc}
  \usepackage[utf8]{inputenc}
\else % if luatex or xelatex
  \ifxetex
    \usepackage{mathspec}
  \else
    \usepackage{fontspec}
  \fi
  \defaultfontfeatures{Ligatures=TeX,Scale=MatchLowercase}
\fi
% use upquote if available, for straight quotes in verbatim environments
\IfFileExists{upquote.sty}{\usepackage{upquote}}{}
% use microtype if available
\IfFileExists{microtype.sty}{%
\usepackage{microtype}
\UseMicrotypeSet[protrusion]{basicmath} % disable protrusion for tt fonts
}{}
\usepackage[margin=1in]{geometry}
\usepackage{hyperref}
\hypersetup{unicode=true,
            pdftitle={ASA DataFest 2019},
            pdfauthor={Jialun Lyu},
            pdfborder={0 0 0},
            breaklinks=true}
\urlstyle{same}  % don't use monospace font for urls
\usepackage{color}
\usepackage{fancyvrb}
\newcommand{\VerbBar}{|}
\newcommand{\VERB}{\Verb[commandchars=\\\{\}]}
\DefineVerbatimEnvironment{Highlighting}{Verbatim}{commandchars=\\\{\}}
% Add ',fontsize=\small' for more characters per line
\usepackage{framed}
\definecolor{shadecolor}{RGB}{248,248,248}
\newenvironment{Shaded}{\begin{snugshade}}{\end{snugshade}}
\newcommand{\AlertTok}[1]{\textcolor[rgb]{0.94,0.16,0.16}{#1}}
\newcommand{\AnnotationTok}[1]{\textcolor[rgb]{0.56,0.35,0.01}{\textbf{\textit{#1}}}}
\newcommand{\AttributeTok}[1]{\textcolor[rgb]{0.77,0.63,0.00}{#1}}
\newcommand{\BaseNTok}[1]{\textcolor[rgb]{0.00,0.00,0.81}{#1}}
\newcommand{\BuiltInTok}[1]{#1}
\newcommand{\CharTok}[1]{\textcolor[rgb]{0.31,0.60,0.02}{#1}}
\newcommand{\CommentTok}[1]{\textcolor[rgb]{0.56,0.35,0.01}{\textit{#1}}}
\newcommand{\CommentVarTok}[1]{\textcolor[rgb]{0.56,0.35,0.01}{\textbf{\textit{#1}}}}
\newcommand{\ConstantTok}[1]{\textcolor[rgb]{0.00,0.00,0.00}{#1}}
\newcommand{\ControlFlowTok}[1]{\textcolor[rgb]{0.13,0.29,0.53}{\textbf{#1}}}
\newcommand{\DataTypeTok}[1]{\textcolor[rgb]{0.13,0.29,0.53}{#1}}
\newcommand{\DecValTok}[1]{\textcolor[rgb]{0.00,0.00,0.81}{#1}}
\newcommand{\DocumentationTok}[1]{\textcolor[rgb]{0.56,0.35,0.01}{\textbf{\textit{#1}}}}
\newcommand{\ErrorTok}[1]{\textcolor[rgb]{0.64,0.00,0.00}{\textbf{#1}}}
\newcommand{\ExtensionTok}[1]{#1}
\newcommand{\FloatTok}[1]{\textcolor[rgb]{0.00,0.00,0.81}{#1}}
\newcommand{\FunctionTok}[1]{\textcolor[rgb]{0.00,0.00,0.00}{#1}}
\newcommand{\ImportTok}[1]{#1}
\newcommand{\InformationTok}[1]{\textcolor[rgb]{0.56,0.35,0.01}{\textbf{\textit{#1}}}}
\newcommand{\KeywordTok}[1]{\textcolor[rgb]{0.13,0.29,0.53}{\textbf{#1}}}
\newcommand{\NormalTok}[1]{#1}
\newcommand{\OperatorTok}[1]{\textcolor[rgb]{0.81,0.36,0.00}{\textbf{#1}}}
\newcommand{\OtherTok}[1]{\textcolor[rgb]{0.56,0.35,0.01}{#1}}
\newcommand{\PreprocessorTok}[1]{\textcolor[rgb]{0.56,0.35,0.01}{\textit{#1}}}
\newcommand{\RegionMarkerTok}[1]{#1}
\newcommand{\SpecialCharTok}[1]{\textcolor[rgb]{0.00,0.00,0.00}{#1}}
\newcommand{\SpecialStringTok}[1]{\textcolor[rgb]{0.31,0.60,0.02}{#1}}
\newcommand{\StringTok}[1]{\textcolor[rgb]{0.31,0.60,0.02}{#1}}
\newcommand{\VariableTok}[1]{\textcolor[rgb]{0.00,0.00,0.00}{#1}}
\newcommand{\VerbatimStringTok}[1]{\textcolor[rgb]{0.31,0.60,0.02}{#1}}
\newcommand{\WarningTok}[1]{\textcolor[rgb]{0.56,0.35,0.01}{\textbf{\textit{#1}}}}
\usepackage{graphicx,grffile}
\makeatletter
\def\maxwidth{\ifdim\Gin@nat@width>\linewidth\linewidth\else\Gin@nat@width\fi}
\def\maxheight{\ifdim\Gin@nat@height>\textheight\textheight\else\Gin@nat@height\fi}
\makeatother
% Scale images if necessary, so that they will not overflow the page
% margins by default, and it is still possible to overwrite the defaults
% using explicit options in \includegraphics[width, height, ...]{}
\setkeys{Gin}{width=\maxwidth,height=\maxheight,keepaspectratio}
\IfFileExists{parskip.sty}{%
\usepackage{parskip}
}{% else
\setlength{\parindent}{0pt}
\setlength{\parskip}{6pt plus 2pt minus 1pt}
}
\setlength{\emergencystretch}{3em}  % prevent overfull lines
\providecommand{\tightlist}{%
  \setlength{\itemsep}{0pt}\setlength{\parskip}{0pt}}
\setcounter{secnumdepth}{0}
% Redefines (sub)paragraphs to behave more like sections
\ifx\paragraph\undefined\else
\let\oldparagraph\paragraph
\renewcommand{\paragraph}[1]{\oldparagraph{#1}\mbox{}}
\fi
\ifx\subparagraph\undefined\else
\let\oldsubparagraph\subparagraph
\renewcommand{\subparagraph}[1]{\oldsubparagraph{#1}\mbox{}}
\fi

%%% Use protect on footnotes to avoid problems with footnotes in titles
\let\rmarkdownfootnote\footnote%
\def\footnote{\protect\rmarkdownfootnote}

%%% Change title format to be more compact
\usepackage{titling}

% Create subtitle command for use in maketitle
\providecommand{\subtitle}[1]{
  \posttitle{
    \begin{center}\large#1\end{center}
    }
}

\setlength{\droptitle}{-2em}

  \title{ASA DataFest 2019}
    \pretitle{\vspace{\droptitle}\centering\huge}
  \posttitle{\par}
    \author{Jialun Lyu}
    \preauthor{\centering\large\emph}
  \postauthor{\par}
      \predate{\centering\large\emph}
  \postdate{\par}
    \date{May 15, 2019}


\begin{document}
\maketitle

\hypertarget{introduction}{%
\subsection{Introduction}\label{introduction}}

In our project, we first tried to examine whether the current overall
measurement of fatigue, ``Monotoring Score'' can be improved or not. To
do this, we noticed that if ``Monitoring Score'' is indeed a good
overall measure of fatigue, then for groups of players who believe they
were performing better than usual in the current game, we should expect
their ``Monitoring Score'' are on average larger than that of groups of
players who think they didnot perform as well as usual in the games.

We can use two sample t-test as follows:

\begin{verbatim}
## 
##  F test to compare two variances
## 
## data:  ANOVA$MonitoringScore[ANOVA$BestOutOfMyself == "Not at all"] and ANOVA$MonitoringScore[ANOVA$BestOutOfMyself == "Absolutely"]
## F = 0.77258, num df = 10, denom df = 46, p-value = 0.692
## alternative hypothesis: true ratio of variances is not equal to 1
## 95 percent confidence interval:
##  0.3299648 2.4979691
## sample estimates:
## ratio of variances 
##          0.7725843
\end{verbatim}

Since the two groups have variances not significantly different from
each other, we can use pooled variance two sample t-tests.

\begin{Shaded}
\begin{Highlighting}[]
\KeywordTok{t.test}\NormalTok{(ANOVA}\OperatorTok{$}\NormalTok{MonitoringScore[ANOVA}\OperatorTok{$}\NormalTok{BestOutOfMyself}\OperatorTok{==}\StringTok{"Not at all"}\NormalTok{],}
\NormalTok{       ANOVA}\OperatorTok{$}\NormalTok{MonitoringScore[ANOVA}\OperatorTok{$}\NormalTok{BestOutOfMyself}\OperatorTok{==}\StringTok{"Absolutely"}\NormalTok{], }\DataTypeTok{var.equal =}\NormalTok{ T)}
\end{Highlighting}
\end{Shaded}

\begin{verbatim}
## 
##  Two Sample t-test
## 
## data:  ANOVA$MonitoringScore[ANOVA$BestOutOfMyself == "Not at all"] and ANOVA$MonitoringScore[ANOVA$BestOutOfMyself == "Absolutely"]
## t = -0.97797, df = 56, p-value = 0.3323
## alternative hypothesis: true difference in means is not equal to 0
## 95 percent confidence interval:
##  -3.354976  1.153816
## sample estimates:
## mean of x mean of y 
##  16.09091  17.19149
\end{verbatim}

\includegraphics{Project_files/figure-latex/unnamed-chunk-4-1.pdf}

We can then check the assumptions of the two sample t-tests.

\includegraphics{Project_files/figure-latex/unnamed-chunk-5-1.pdf}

It seems that normality assumption is satisfied.

It seems that based on the boxplots, these two groups seem to have
similar ``Monitoring Scores'' instead. Does that mean these five
subjective measures of fatigues are totally useless? Maybe not! Since we
know that even though those measures are subjective and may even be
biased in some ways, they are still very good proxy to the real physical
fatigue levels of players. Therefore, we just need to find a more
efficient way to use this information.

To do that, we decided to apply PCA on these five standardized
subjective features, and the reason that we standardized these features
is to take in to account the individual's variations. Using R, we found
all of the principal component vectors, and we noticed that the first
principal component vector alone explains the most of variability among
these features.

\begin{Shaded}
\begin{Highlighting}[]
\NormalTok{X <-}\StringTok{ }\KeywordTok{select}\NormalTok{(PlayerinGame,}\StringTok{"Fatigue"}\NormalTok{,}\StringTok{"Soreness"}\NormalTok{,}\StringTok{"Desire"}\NormalTok{,}\StringTok{"Irritability"}\NormalTok{,}\StringTok{"SleepQuality"}\NormalTok{,}\StringTok{"Mspeed"}\NormalTok{,}\StringTok{"Macimpulse"}\NormalTok{,}\StringTok{"Macload"}\NormalTok{)}
\NormalTok{PCA <-}\StringTok{ }\KeywordTok{prcomp}\NormalTok{(X)}
\KeywordTok{summary}\NormalTok{(PCA)}
\end{Highlighting}
\end{Shaded}

\begin{verbatim}
## Importance of components:
##                           PC1    PC2    PC3    PC4     PC5     PC6     PC7
## Standard deviation     1.7949 1.0925 0.9481 0.8465 0.61243 0.55226 0.10104
## Proportion of Variance 0.4794 0.1776 0.1338 0.1066 0.05581 0.04538 0.00152
## Cumulative Proportion  0.4794 0.6569 0.7907 0.8973 0.95310 0.99848 1.00000
##                            PC8
## Standard deviation     0.00151
## Proportion of Variance 0.00000
## Cumulative Proportion  1.00000
\end{verbatim}

\begin{Shaded}
\begin{Highlighting}[]
\KeywordTok{biplot}\NormalTok{(PCA}\OperatorTok{$}\NormalTok{x,PCA}\OperatorTok{$}\NormalTok{rotation[,}\DecValTok{1}\OperatorTok{:}\DecValTok{5}\NormalTok{])}
\end{Highlighting}
\end{Shaded}

\includegraphics{Project_files/figure-latex/Boxplots-1.pdf}

\begin{Shaded}
\begin{Highlighting}[]
\NormalTok{X1 <-}\StringTok{ }\KeywordTok{mutate}\NormalTok{(X,}\DataTypeTok{PC1=}\KeywordTok{as.matrix}\NormalTok{(X)}\OperatorTok\NormalTok{PCA}\OperatorTok{$}\NormalTok{rotation[,}\DecValTok{1}\NormalTok{])}
\NormalTok{X1 <-}\StringTok{ }\KeywordTok{mutate}\NormalTok{(X1,}\DataTypeTok{PC2=}\KeywordTok{as.matrix}\NormalTok{(X)}\OperatorTok\NormalTok{PCA}\OperatorTok{$}\NormalTok{rotation[,}\DecValTok{2}\NormalTok{])}
\NormalTok{X1}\OperatorTok{$}\NormalTok{SleepHours <-}\StringTok{ }\NormalTok{PlayerinGame}\OperatorTok{$}\NormalTok{SleepHours}
\NormalTok{PlayerinGame}\OperatorTok{$}\NormalTok{OurMeasure <-}\StringTok{ }\NormalTok{X1}\OperatorTok{$}\NormalTok{PC1}
\end{Highlighting}
\end{Shaded}

Therefore, we will assign the components of first loading vector as
weights to these five features and compute a weighted sum of them, and
that weighted sum will be our new measure of fatigue called ``Sscore''.
Now, we claim that this ``Sscore'' computed using PCA, will be much
better than ``Monitoring Scores'', since first principal component
vector by definition is the best direction in terms of efficiently
utilizing all the information in the feature space. To confirm that, we
can do a pooled t-test between Sscores of players with higher
self-rating and Sscores of players with lower self-rating, since all the
diagnostics such as normal qqplots and F test of variance, showed that
our pooled t-test is valid. Consequently, we find that now we have
strong evidence to say that these two groups are on average with
different Sscores which we failed to conclude when we used ``Monitoring
Scores''. This result strongly suggests that we should think about using
our ``Sscore'' to measure the overall fatigue instead of the original
Monitoring Scores.


\end{document}
